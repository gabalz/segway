\documentclass[a4paper]{article}
\usepackage[top=2cm,left=2cm,right=2cm]{geometry}
\usepackage{graphicx}
\usepackage{caption}
\usepackage{amsfonts}
\usepackage{amsmath}
\usepackage{amsthm}
\usepackage{amssymb}
\usepackage[svgnames]{xcolor}
\usepackage[colorlinks=true,
            linkcolor=blue,
            citecolor=blue,
            urlcolor=blue]{hyperref}
\usepackage{todonotes}

\newcommand{\R}{\mathbb{R}} % set of real numbers
\newcommand{\enorm}[1]{\lVert#1\rVert_2} % Euclidean norm

\renewcommand{\vec}[1]{\mathbf{#1}} % vector
\newcommand{\vx}{\vec{x}}
\newcommand{\vp}{\vec{p}}
\newcommand{\vv}{\vec{v}}

\newcommand{\hH}{\widehat{H}}
\newcommand{\hh}{\widehat{h}}
\newcommand{\hf}{\widehat{f}}
\newcommand{\hG}{\widehat{G}}

\newcommand{\I}{\textbf{I}} % identity matrix
\newcommand{\Z}{\vec{0}} % zero matrix

\newcommand{\pitch}{\psi}
\newcommand{\yaw}{\phi}
\newcommand{\roll}{\theta}

%------------------------------------------------------------------------------

\begin{document}

\title{Segway Model}
\maketitle

\section{Parameters and notation}

Our notation is similar to \cite{nxtway-gs}.
\begin{center}
\begin{tabular}{l@{\ =\ }c@{\quad:\quad}l}
    $g$ & $9.80665$ & gravitational acceleration ($m/s^2$) \\
    \hline
    $m$ & $0.0165 \,/\, 0.0295$ & NXT2/RCX wheel weight ($kg$) \\
    $R$ & $0.0216 \,/\, 0.0408$ & NXT2/RCX wheel radius ($m$) \\
    $w$ & $0.022 \,/\, 0.015$ & NXT2/RCX wheel width ($m$) \\
    $J_w$ & $m R^2 / 2$ & wheel inertia moment ($kg \cdot m^2$) \\
    \hline
    $M$ & $0.55$ & body weight ($kg$) \\
    $W$ & $0.15$ & body width ($m$) \\
    $D$ & $0.045$ & body depth ($m$) \\
    $H$ & $0.158$ & body height ($m$) \\
    $L$ & $H/2$ & center of mass distance from the wheel axle ($m$) \\
    $J_\pitch$ & $0.00805$
                & body pitch moment of inertia ($kg \cdot m^2$) \\
    $J_\yaw$ & $0.005$
                    & body yaw moment of inertia ($kg \cdot m^2$) \\
    $\pitch_0$ & $-2$ & balanced pitch deviation from upright position (deg) \\
    \hline
    $B$ & $0.18$ & DC motor constant ($Nm \cdot s/rad$) \\
    $K$ & $20$ & DC motor constant ($Nm / V$) \\
    \hline
    $\pitch$ & & body pitch angle ($rad$) \\
    $\roll_{l/r}$ & $\roll \mp W \yaw / (2 R)$
                        & left/right wheel roll angles ($rad$) \\
    $\roll$ & $(\roll_l + \roll_r)/2$ & axle midpoint roll angle ($rad$) \\
    $\yaw$ & $R \, (\roll_r - \roll_l) / W$ & body yaw angle ($rad$) \\
    $\vp_{l/r}$ & $[x_{l/r} \,\, y_{l/r} \,\, R]^T$
                    & left/right wheel positions ($m$) \\
    $\vp_m$ & $[x_m \,\, y_m \,\, R]^T$ & axle midpoint position ($m$) \\
    $\vp_c$ & $[x_c \,\, y_c \,\, z_c]^T$ 
                    & body center of mass position ($m$) \\
    $v_{l/r}$ & & left/right DC motor control power (V) \\
\end{tabular}
\end{center}

\subsection{Robot position}

The position of the center of mass and its derivative can be expressed as
\begin{equation}
    \vp_c = \vp_m + L \left[ \begin{array}{c}
                                \sin\pitch \cos\yaw \\
                                \sin\pitch \sin\yaw \\
                                \cos\pitch \\
                             \end{array} \right]
    \quad , \quad
    \dot{\vp}_c = \dot{\vp}_m
    + L \left[ \begin{array}{c}
               \dot\pitch \cos\pitch \cos\yaw - \dot\yaw \sin\pitch \sin\yaw \\
               \dot\pitch \cos\pitch \sin\yaw + \dot\yaw \sin\pitch \cos\yaw \\
               - \dot\pitch \sin\pitch
               \end{array} \right]
    \,,
\end{equation}
where $\vp_m = (\vp_l + \vp_r) / 2$.
Furthermore, if the $\yaw$ is constant and the wheels are rotating
$\delta\roll_l$ and $\delta\roll_r$ radians in $\delta t$ seconds, then
the wheel positions are changing by
\begin{equation}
    \delta \vp_l = R \, \delta\roll_l \left[ \begin{array}{c}
                                                \cos\yaw \\
                                                \sin\yaw \\
                                                0 \\
                                             \end{array} \right]
    \quad , \quad
    \delta \vp_r = R \, \delta\roll_r \left[ \begin{array}{c}
                                                \cos\yaw \\
                                                \sin\yaw \\
                                                0 \\
                                             \end{array} \right]
    \,.
\end{equation}
As we take $\delta t \rightarrow 0$, $\yaw$ becomes constant on
this infinitely short $\delta t$ interval, and so we obtain
\begin{equation}
    \dot{\vp}_l = R \, \dot\roll_l \left[ \begin{array}{c}
                                    \cos\yaw \\
                                    \sin\yaw \\
                                    0 \\
                                   \end{array} \right]
    \quad , \quad
    \dot{\vp}_r = R \, \dot\roll_r \left[ \begin{array}{c}
                                    \cos\yaw \\
                                    \sin\yaw \\
                                    0 \\
                                   \end{array} \right]
    \quad , \quad
    \dot{\vp}_m = R \, \dot\roll \left[ \begin{array}{c}
                                    \cos\yaw \\
                                    \sin\yaw \\
                                    0 \\
                                   \end{array} \right]
    \,.
\end{equation}

\subsection{Pitch equilibrium}

This model assumes that the center of mass is of the robot body is just
above the axle at the middle when the robot is in equilibrium. In real it
is not quite true, so one can shift the pitch angle by a constant $\pitch_0$
and use $\pitch - \pitch_0$ in the model.

But notice that moving the body center of mass also changes the body moment of 
inertias, $J_\pitch$ and $J_\yaw$, so these quantities should be identified 
for a fixed $\pitch_0$.

\section{Motion equations}

The system is divided into two subsystems: the driven segway robot
and the driving DC motors.

\subsection{Segway robot dynamics}

The Lagrangian of the system (without the DC motor) is
\begin{equation}
    L = E_t + E_r - E_p \,,
\end{equation}
where $E_t$ is the translational kinetic energy, $E_r$ is the rotational 
kinetic energy and $E_p$ is the potential energy defined as
\begin{equation} \begin{split}
    E_t &= \frac{m}{2} \big( \dot{x}_l^2 + \dot{y}_l^2 \big)
         + \frac{m}{2} \big( \dot{x}_r^2 + \dot{y}_r^2 \big)
         + \frac{M}{2} \big( \dot{x}_c^2 + \dot{y}_c^2 + \dot{z}_c^2 \big)
        \,, \\
    E_r &= \frac{J_w}{2} \dot{\roll}_l^2 + \frac{J_w}{2} \dot{\roll}_r^2
         + \frac{J_\pitch}{2} \dot{\pitch}^2 + \frac{J_\yaw}{2} \dot{\yaw}^2
        \,, \\
    E_p &= m g R + m g R + M g z_c
        \,.
\end{split} \end{equation}
Then to find the equations of motion with respect to $\pitch$, $\yaw$ and
$\roll$, we write the Euler-Lagrange equations for the $F_\pitch$, $F_\yaw$,
$F_\roll$ generalized forces (see appendix \ref{sec:euler-lagrange-deriv}) as
\begin{equation} \begin{split}
    F_\pitch &= \frac{d}{dt}\left(\frac{\partial L}{\partial\dot\pitch}\right)
              - \frac{\partial L}{\partial\pitch}
              = \big( M L^2 + J_\pitch \big) \ddot\pitch
              + \big( M L R \cos\pitch \big) \ddot\roll
              - M L^2 \dot\yaw^2 \sin\pitch \cos\pitch
              - M g L \sin\pitch
    \,, \\
    F_\roll &= \frac{d}{dt}\left(\frac{\partial L}{\partial\dot\roll}\right)
             - \frac{\partial L}{\partial\roll}
             = \bigg( \frac{M L R}{2} \cos\pitch \bigg) \ddot\pitch
             + \bigg( m R^2 + J_w + \frac{M R^2}{2} \bigg) \ddot\roll
             - \frac{M L R}{2} \dot\pitch^2 \sin\pitch
    \,, \\
    F_\yaw &= \frac{d}{dt}\left(\frac{\partial L}{\partial\dot\yaw}\right)
            - \frac{\partial L}{\partial\yaw}
            = \bigg( m R^2 + J_w
                   + \frac{2 R^2}{W^2}
                     \big( M L^2 \sin^2\pitch + J_\yaw \big)
              \bigg) \ddot\yaw
            + \frac{2 M R^2 L^2}{W^2}
              \dot\yaw \dot\pitch \sin\pitch \cos\pitch
    \,.
\end{split} \end{equation}

\subsection{DC motor dynamics}

A detailed description of a DC motor model can be found in
\cite{robot-modeling}. However, now we will use a simplified model, which
captures only that the torque is damped by the shaft's rotational velocity
and proportional to the applied power. So
\begin{equation}
    F_{\roll_{l/r}} = -B (\dot\roll_{l/r} - \dot\pitch)
                       + K v_{l/r}
    \,,
\end{equation}
where $B$ ($Nm \cdot rad/s$) and $K$ ($Nm / V$) are DC motor constants.
Then we can express the generalized forces as
\begin{equation} \begin{split}
    F_\pitch &= -(F_{\roll_l} + F_{\roll_r})
              = 2 B (\dot\roll - \dot\pitch) - K(v_l + v_r)
    \,, \\
    F_\roll &= \frac{1}{2}(F_{\roll_l} + F_{\roll_r})
             = -B (\dot\roll - \dot\pitch) + K \frac{v_l + v_r}{2}
    \,, \\
    F_\yaw &= \frac{R}{W}(F_{\roll_r} - F_{\roll_l})
            = -B \dot\yaw + \frac{R}{W} (v_r - v_l)
    \,.
\end{split} \end{equation}

\subsection{State space form}

By connecting the two subsystems and rearranging the equations, we get
\begin{equation} \begin{split}
\label{eq:motion-raw}
        \Big( M L^2 + J_\psi \Big) \ddot\pitch
      + (M L R \cos\pitch) \ddot\roll
    &=  M L^2 \dot\yaw^2 \sin\pitch \cos\pitch
      + M g L \sin\pitch
      + 2 B \big( \dot\roll - \dot\pitch \big)
      - K \big( v_l + v_r \big)
    \,, \\
        (M L R \cos\pitch) \ddot\pitch
      + \Big( (2 m + M) R^2 + 2 J_w \Big) \ddot\roll
    &=  M L R \dot\pitch^2 \sin\pitch
      - 2 B \big( \dot\roll - \dot\pitch \big)
      + K \big( v_l + v_r \big)
    \,, \\
        \Big( m R^2 W^2 + W^2 J_w
            + 2 R^2 \big( M L^2 \sin^2\pitch + J_\yaw \big)
        \Big) \ddot\yaw
    &=  -2 M R^2 L^2 \dot\yaw \dot\pitch \sin\pitch \cos\pitch
      - B W^2 \dot\yaw + R W K \big( v_r - v_l \big)
    \,.
\end{split} \end{equation}
By introducing the $\vx = [\pitch \,\, \roll \,\, \yaw \,\,
                           \dot\pitch \,\, \dot\roll \,\, \dot\yaw \,\,
                           x_m \,\, y_m]^T$ state
           and the $\vv = [v_l \,\, v_r]^T$ control,
we can write (\ref{eq:motion-raw}) into
\begin{equation}
    H(\vx) \, \dot{\vx} = \hf(\vx) + \hG \, \vv \,,
\end{equation}
where $H : \R^8 \rightarrow \R^{8 \times 8}$,
      $\hf : \R^8 \rightarrow \R^8$,
      $\hG \in \R^{8 \times 2}$,
\begin{equation}
   H(\vx)
   = \left[ \begin{array}{cccc}
     \I_{3 \times 3} & \Z_{3 \times 2}  & \Z_{3 \times 1}  & \Z_{3 \times 2} \\
     \Z_{2 \times 3} & \hH(\vx)         & \Z_{2 \times 1}  & \Z_{1 \times 2} \\
     \Z_{1 \times 3} & \Z_{1 \times 2}  & \hh(\vx)         & \Z_{1 \times 2} \\
     \Z_{2 \times 3} & \Z_{2 \times 2}  & \Z_{2 \times 1}  & \I_{2 \times 2} \\
     \end{array} \right]
   \quad , \quad
   \hf(\vx) = \left[ \begin{array}{c}
                     \dot\pitch \\
                     \dot\roll \\
                     \dot\yaw \\
                     \hf_4(\vx) \\
                     \hf_5(\vx) \\
                     \hf_6(\vx) \\
                     R \, \dot\roll \cos\yaw \\
                     R \, \dot\roll \sin\yaw \\
                     \end{array} \right]
    \quad , \quad
    \hG = K \left[ \begin{array}{cc}
                   0    & 0 \\
                   0    & 0 \\
                   0    & 0 \\
                   -1   & -1 \\
                   1    & 1 \\
                   -R W & R W \\
                   0    & 0 \\
                   0    & 0 \\
                   \end{array} \right]
    \,,
\end{equation}
with
\begin{equation} \begin{split}
    \widehat{H}(\vx) &= \left[ \begin{array}{cc}
                        \hH_{11} & \hH_{12}(\vx) \\
                        \hH_{12}(\vx) & \hH_{22} \\
                        \end{array} \right]
    \qquad , \qquad
    \begin{array}{r@{\,=\,}l}
        \hH_{11} & M L^2 + J_\pitch \\
        \hH_{12}(\vx) & M L R \cos\pitch \\
        \hH_{22} & (2 m + M) R^2 + 2 J_w \\
    \end{array}
    \,, \\
    \hh(\vx) &= m R^2 W^2 + W^2 J_w
              + 2 R^2 \big( M L^2 \sin^2\pitch + J_\yaw \big)
    \,, \\
    \hf_4(\vx) &= M L^2 \dot\yaw^2 \sin\pitch \cos\pitch
                + M g L \sin\pitch
                + 2 B \big( \dot\roll - \dot\pitch \big)
    \,, \\
    \hf_5(\vx) &= M L R \dot\pitch^2 \sin\pitch
                - 2 B \big( \dot\roll - \dot\pitch \big)
    \,, \\
    \hf_6(\vx) &= -2 M R^2 L^2 \dot\yaw \dot\pitch \sin\pitch \cos\pitch
                - B W^2 \dot\yaw
    \,.
\end{split} \end{equation}
Notice that $\det\hH(\vx) > 0$ for all $\vx$ because
\begin{equation} \begin{split}
    \det\hH(\vx) &= \big( M L^2 + J_\psi \big)
                    \Big( (2m + M) R^2 + 2 J_w \Big)
                  - \big( M L R \cos\pitch \big)^2
                 \\
                 &> M^2 L^2 R^2
                  - M^2 L^2 R^2 \cos^2\pitch
                  = M^2 L^2 R^2 \big( 1 - \cos^2\pitch \big)
                  \ge 0
                  \,.
\end{split} \end{equation}
Furthermore, for all $\vx$, we have $\hh(\vx) > 0$ and so $\det H(\vx) > 0$.
Hence, $\exists H(\vx)^{-1}$ for all $\vx$, and the state space form becomes
\begin{equation}
    \dot{\vx} = H(\vx)^{-1} \Big(\hf(\vx) + \hG \, \vv\Big)
              = H(\vx)^{-1} \hf(\vx) + H(\vx)^{-1} \hG \, \vv
              = f(\vx) + G(\vx) \, \vv
              \,,
\end{equation}
where
\begin{equation}
    f(\vx) = \left[ \begin{array}{c}
                    \dot\pitch \\
                    \dot\roll \\
                    \dot\yaw \\
                    f_4(\vx) \\
                    f_5(\vx) \\
                    f_6(\vx) \\
                    R \dot\roll \cos\yaw \\
                    R \dot\roll \sin\yaw \\
                    \end{array} \right]
    \qquad , \qquad
    G(\vx) = \left[ \begin{array}{cc}
                    0 & 0 \\
                    0 & 0 \\
                    0 & 0 \\
                    g_4(\vx) & g_4(\vx) \\
                    g_5(\vx) & g_5(\vx) \\
                    -g_6(\vx) & g_6(\vx) \\
                    0 & 0 \\
                    0 & 0 \\
                    \end{array} \right]
    \,,
\end{equation}
with
\begin{equation} \begin{aligned}
    f_4(\vx) &= \frac{\hH_{22}\hf_4(\vx) - \hH_{12}(\vx)\hf_5(\vx)}
                     {\hH_{11}\hH_{22} - \hH_{12}(\vx)^2}
    \quad &, \quad
    g_4(\vx) &= \frac{-K\big(\hH_{22} + \hH_{12}(\vx)\big)}
                     {\hH_{11}\hH_{22} - \hH_{12}(\vx)^2}
    \,, \\
    f_5(\vx) &= \frac{-\hH_{12}(\vx)\hf_4(\vx) + \hH_{11}\hf_5(\vx)}
                     {\hH_{11}\hH_{22} - \hH_{12}(\vx)^2}
    \quad &, \quad
    g_5(\vx) &= \frac{K\big(\hH_{12}(\vx) + \hH_{11}\big)}
                     {\hH_{11}\hH_{22} - \hH_{12}(\vx)^2}
    \,, \\
    f_6(\vx) &= \frac{\hf_6(\vx)}
                     {\hh(\vx)}
    \quad &, \quad
    g_6(\vx) &= \frac{K R W}{\hh(\vx)}
    \,.
\end{aligned} \end{equation}

\appendix

\section{Derivation of the Euler-Lagrange equations}
\label{sec:euler-lagrange-deriv}

Notice that
\begin{equation} \begin{split}
\label{eq:dot-xy}
    \dot{x}_c \cos\yaw + \dot{y}_c \sin\yaw
        &= \dot{x}_m \cos\yaw + \dot{y}_m \sin\yaw
         + L \dot\pitch \cos\pitch \big( \cos^2\yaw + \sin^2\yaw \big)
         = R \dot\roll + L \dot\pitch \cos\pitch
    \,, \\
    \dot{x}_c \sin\yaw - \dot{y}_c \cos\yaw
        &= \dot{x}_m \sin\yaw - \dot{y}_m \cos\yaw
         - L \dot\yaw \sin\pitch \big( \sin^2\yaw + \cos^2\yaw \big)
         = - L \dot\yaw \sin\pitch
    \,.
\end{split} \end{equation}
Then
\begin{equation} \begin{split}
    F_\pitch
    &= \frac{d}{dt} \left( \frac{\partial L}{\partial \dot\pitch} \right)
     - \frac{\partial L}{\partial \pitch}
    \\
    &= \frac{d}{dt} \Big( M L \big( \dot{x}_c \cos\pitch \cos\yaw
                                  + \dot{y}_c \cos\pitch \sin\yaw
                                  - \dot{z}_c \sin\pitch
                              \big)
                        + J_\pitch \dot\pitch
                    \Big)
    \\ &\quad + M L \Big( \dot{x}_c \big( \dot\pitch \sin\pitch \cos\yaw
                                        + \dot\yaw \cos\pitch \sin\yaw
                                    \big)
                        + \dot{y}_c \big( \dot\pitch \sin\pitch \sin\yaw
                                        - \dot\yaw \cos\pitch \cos\yaw
                                    \big)
                        + \dot{z}_c \dot\pitch \cos\pitch
                        - g \sin\pitch
                    \Big)
    \\
    &= \frac{d}{dt} \Big( M L \cos\pitch \big( \dot{x}_c \cos\yaw
                                             + \dot{y}_c \sin\yaw
                                         \big)
                        + M L^2 \dot\pitch \sin^2\pitch
                        + J_\pitch \dot\pitch
                    \Big)
    \\ &\quad + M L \Big( \dot\pitch \sin\pitch \big( \dot{x}_c \cos\yaw
                                                    + \dot{y}_c \sin\yaw
                                                \big)
                        + \dot\yaw \cos\pitch \big( \dot{x}_c \sin\yaw
                                                  - \dot{y}_c \cos\yaw
                                              \big)
                        - L \, \dot\pitch^2 \sin\pitch \cos\pitch
                        - g \sin\pitch
                    \Big)
    \\
    &\hspace{-1.5mm} \stackrel{(\ref{eq:dot-xy})}{=}
        \frac{d}{dt} \Big( M L R \, \dot\roll \cos\pitch
                         + M L^2 \, \dot\pitch
                         + J_\pitch \dot\pitch
                     \Big)
    \\ &\quad + M L \big( R \, \dot\roll \, \dot\pitch \sin\pitch
                        + L \, \dot\pitch^2 \sin\pitch \cos\pitch
                        - L \, \dot\yaw^2 \cos\pitch \sin\pitch
                        - L \, \dot\pitch^2 \sin\pitch \cos\pitch
                        - g \sin\pitch
                    \big)
    \\
    &= M L R \, \ddot\roll \cos\pitch
     - M L R \, \dot\roll \, \dot\pitch \sin\pitch
     + \big( M L^2 + J_\pitch \big) \ddot\pitch
     + M L \big( R \, \dot\roll \, \dot\pitch \sin\pitch
               - L \, \dot\yaw^2 \sin\pitch \cos\pitch
               - g \sin\pitch
           \big)
    \\
    &= \big( M L^2 + J_\pitch \big) \ddot\pitch
     + \big( M L R \cos\pitch \big) \ddot\roll
     - M L^2 \, \dot\yaw^2 \sin\pitch \cos\pitch
     - M g L \sin\pitch
    \,.
\end{split} \end{equation}
Furthermore, notice that $\partial\yaw / \partial\roll_{l/r} = \mp R/W$ and
\begin{equation} \begin{split}
\label{eq:xy-lr}
    \dot{x}_{l/r} \cos\yaw + \dot{y}_{l/r} \sin\yaw
        &= R \, \dot\roll_{l/r} \cos^2\yaw + R \, \dot\roll_{l/r} \sin^2\yaw
         = R \, \dot\roll_{l/r}
    \,, \\
    \dot{x}_{l/r} \sin\yaw - \dot{y}_{l/r} \cos\yaw
        &= R \, \dot\roll_{l/r} \cos\yaw \sin\yaw
         - R \, \dot\roll_{l/r} \sin\yaw \cos\yaw
         = 0
    \,, \\
    \dot{x}_c \cos\yaw + \dot{y}_c \sin\yaw
        &= R \, \dot\roll \cos^2\yaw 
         + L \big( \dot\pitch \cos\pitch \cos^2\yaw
                 - \dot\yaw \sin\pitch \sin\yaw \cos\yaw
             \big)
        \\ &\quad
         + R \, \dot\roll \sin^2\yaw 
         + L \big( \dot\pitch \cos\pitch \sin^2\yaw
                 + \dot\yaw \sin\pitch \cos\yaw \sin\yaw
             \big)
         = R \, \dot\roll + L \dot\pitch \cos\pitch
    \,, \\
    \dot{x}_c \sin\yaw - \dot{y}_c \cos\yaw
        &= R \, \dot\roll \cos\yaw \sin\yaw
         + L \big( \dot\pitch \cos\pitch \cos\yaw \sin\yaw
                 - \dot\yaw \sin\pitch \sin^2\yaw
             \big)
        \\ &\quad
         - R \, \dot\roll \sin\yaw \cos\yaw
         - L \big( \dot\pitch \cos\pitch \sin\yaw \cos\yaw
                 + \dot\yaw \sin\pitch \cos^2\yaw
             \big)
         = - L \dot\yaw \sin\pitch
    \,.
\end{split} \end{equation}
Then the forces of the left/right axle endpoints can be expressed as
\begin{equation} \begin{split}
    F_{\roll_{l/r}}
    &= \frac{d}{dt} \left( \frac{\partial L}{\partial \dot\roll_{l/r}} \right)
     - \frac{\partial L}{\partial \roll_{l/r}}
    \\
    &= \frac{d}{dt} \bigg( m R \big( \dot{x}_{l/r} \cos\yaw
                                   + \dot{y}_{l/r} \sin\yaw
                               \big)
    \\ &\hspace{2cm}
                         + \frac{M R}{2}
                           \Big[ \dot{x}_c
                                 \Big( \cos\yaw
                                   \pm \frac{2 L}{W} \sin\pitch \sin\yaw
                                 \Big)
                               + \dot{y}_c
                                 \Big( \sin\yaw
                                   \mp \frac{2 L}{W} \sin\pitch \cos\yaw
                                 \Big)
                           \Big]
                          + J_w \dot\roll_{l/r}
                          \mp \frac{R}{W} J_\yaw \dot\yaw
                      \bigg)
    \\ &\quad \mp \frac{m R^2}{W} \big( \dot{x}_l \dot\roll_l \sin\yaw
                                      - \dot{y}_l \dot\roll_l \cos\yaw
                                      + \dot{x}_r \dot\roll_r \sin\yaw
                                      - \dot{x}_r \dot\roll_r \cos\yaw
                                \big)
    \\ &\quad \mp \frac{M R}{W}
                  \Big( \dot{x}_c \big[ R \, \dot\roll \sin\yaw
                                      + L \big( \dot\pitch \cos\pitch \sin\yaw
                                              + \dot\yaw \sin\pitch \cos\yaw
                                          \big)
                                  \big]
    \\ &\hspace{6cm}
                      + \dot{y}_c \big[ - R \, \dot\roll \cos\yaw
                                      + L \big( -\dot\pitch \cos\pitch \cos\yaw
                                              + \dot\yaw \sin\pitch \sin\yaw        
                                          \big)
                                  \big]
                  \Big)
    \\
    &\hspace{-1.5mm} \stackrel{(\ref{eq:xy-lr})}{=}
        \frac{d}{dt} \bigg( m R^2 \dot\roll_l
                          + \frac{M R}{2} \big( R \, \dot\roll
                                              + L \dot\pitch \cos\pitch
                                          \big)
                          \mp \frac{M R L^2}{W} \dot\yaw \sin^2\pitch
                          + J_w \dot\roll_{l/r}
                          \mp \frac{R}{W} J_\yaw \dot\yaw
                     \bigg)
    \\ &\quad \mp 0
              \mp \frac{M R}{W} \Big( R \dot\roll
                                    + L \dot\pitch \cos\pitch
                                \Big)
                                \big( - L \dot\yaw \sin\pitch \big)
              \pm \frac{M R L}{W} \dot\yaw \sin\pitch
                                  \big( R \, \dot\roll
                                      + L \dot\pitch \cos\pitch \big)
    \\
    &= m R^2 \ddot\roll_{l/r}
     + \frac{M R}{2} \big( R \, \ddot\roll
                         + L \ddot\pitch \cos\pitch
                         - L \dot\pitch^2 \sin\pitch
                     \big)
     \mp \frac{M R L^2}{W} \big( \ddot\yaw \sin^2\pitch
                               + 2 \dot\yaw \dot\pitch \sin\pitch \cos\pitch
                           \big)
     \\ &\hspace{12cm}
     + J_w \ddot\roll_{l/r}
     \mp \frac{R}{W} J_\yaw \ddot\yaw
     \\
     &= \Big( m R^2 + J_w \Big) \ddot\roll_{l/r}
      + \frac{M R^2}{2} \ddot\roll
      + \bigg( \frac{M R L}{2} \cos\pitch \bigg) \ddot\pitch
      - \frac{M R L}{2} \dot\pitch^2 \sin\pitch
     \\ &\hspace{6cm}
      \mp \frac{R}{W}
          \bigg( M L^2 \sin^2\pitch + J_\yaw \bigg) \ddot\yaw
      \mp \frac{2 M R L^2}{W} \dot\yaw \dot\pitch \sin\pitch \cos\pitch
     \,.
\end{split} \end{equation}
By using $F_{\roll_l}$ and $F_{\roll_r}$, we can compute
the $F_\roll$, $F_\yaw$ generalized forces as
\begin{equation} \begin{split}
    F_\roll &= F_{\roll_l} \frac{\partial \roll}{\partial \roll_l}
             + F_{\roll_r} \frac{\partial \roll}{\partial \roll_r}
             = \frac{F_{\roll_l} + F_{\roll_r}}{2}
            \\
            &= \left( m R^2 + J_w + \frac{M R^2}{2} \right) \ddot\roll
             + \left( \frac{M L R}{2} \cos\pitch \right) \ddot\pitch
             - \frac{M L R}{2} \dot\pitch^2 \sin\pitch
    \,, \\
    F_\yaw &= F_{\roll_l} \frac{\partial \yaw}{\partial \roll_l}
            + F_{\roll_r} \frac{\partial \yaw}{\partial \roll_r}
            = \frac{R}{W} \Big( F_{\roll_r} - F_{\roll_r} \Big)
           \\
           &= \bigg( m R^2 + J_w
                   + \frac{2 R^2}{W^2}
                     \big( M L^2 \sin^2\pitch + J_\yaw \big)
              \bigg) \ddot\yaw
            + \frac{2 M R^2 L^2}{W^2}
              \dot\yaw \dot\pitch \sin\pitch \cos\pitch
    \,.
\end{split} \end{equation}

\begin{thebibliography}{99}

\bibitem{nxtway-gs}
    Yorihisa Yamamoto \\
    \textsl{NXTway-GS (Self-Balancing Two-Wheeled Robot) Controller Design} \\
    \href{http://www.mathworks.com/matlabcentral/fileexchange/19147}
         {http://www.mathworks.com/matlabcentral/fileexchange/19147}

\bibitem{nxt-motor-philo}
    Philippe E. Hurbain \\
    \textsl{NXT Motor Internals} \\
    \href{http://www.philohome.com/nxtmotor/nxtmotor.htm}
         {http://www.philohome.com/nxtmotor/nxtmotor.htm}

\if0
\bibitem{nxt-motor}
    Ryo Watanabe \\
    \textsl{NXT Motor} \\
    \href{http://web.mac.com/ryo_watanabe/iWeb/Ryo%27s%20Holiday/NXT%20Motor.html}
         {http://web.mac.com/ryo\_watanabe/iWeb/Ryo\%27s\%20Holiday/NXT\%20Motor.html}
\fi

\bibitem{robot-modeling}
    Mark W. Spong, Seth Hutchinson, M. Vidyasagar \\
    \textsl{Robot Modeling and Control} \\
    John Wiley \& Sons, 2005

\end{thebibliography}

\end{document}

