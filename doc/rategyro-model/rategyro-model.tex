\documentclass[a4paper]{article}
\usepackage[top=35mm,left=3cm,right=3cm]{geometry}
\usepackage{graphicx}
\usepackage{amsfonts}
\usepackage{amsmath}
\usepackage{color}
\usepackage{verbatim}
\usepackage[colorlinks=true,
            linkcolor=blue,
            citecolor=blue,
            urlcolor=blue]{hyperref}

\newcommand{\todo}[1]{\textcolor{red}{\begin{center}TODO : #1\end{center}}}
\newcommand{\selfref}[1]{\href{#1}{#1}}
\setlength{\parindent}{0cm} % no indentation

\begin{document}

\title{Rate Gyro Model}
\maketitle

\section{Dynamics}

The dynamics of a rate gyroscope  takes the
form\footnote{A. Lawrence, Modern Inertial Technoogy:
              Navigation, Guidance, and Control (2nd edition),
              Mechanical Engineering Series, 1998,
              Chapter 7, Rate Gyro Dynamics (page 100)}
\begin{equation}
\label{eq:dyn:gyro}
    I_a \ddot{\theta} + c \dot{\theta} + K_{tb} \theta = H \omega,
\end{equation}
where $\theta$ is the gimbal angle, $\omega$ is the input angular velocity,
$H$ is the wheel angular momentum, $I_a$ is the gimbal moment of inertia about
the output axis (OA), $c$ is the damping constant about the OA and $K_{tb}$ is
the torsion bar spring constant. The output of the gyro is
\begin{equation}
\label{eq:gyr:output}
    \hat{y} = \theta K_{po} + B,
\end{equation}
where $K_{po}$ is the pickoff sensitivity and $B$ is the measurement bias.
The idea of a rate gyroscope is that by measuring $\hat{y}$ one can get 
information about $\omega$. \\

Assume that the value of the bias $B$ is known
and consider the debiased gyro output $y = \hat{y} - B$.
Clearly, ${y}$ satisfies a differential equation similar to $\theta$:
\begin{eqnarray*}
    {y} &=& K_{po}\theta ,\\
    \dot{{y}} &=& K_{po} \dot{\theta},\\
    \ddot{{y}} &=& K_{po} \ddot{\theta} = \frac{K_{po}}{I_a} 
    (H\omega - c\dot{\theta} - K_{tb}\theta) 
    = \frac{K_{po}}{I_a} \left(H\omega - \frac{c}{K_{po}} 
      \dot{{y}} - \frac{K_{tb}}{K_{po}} y \right),
\end{eqnarray*}
or
\begin{equation}
\label{eq:gyro:harm1}
    \ddot{{y}} + \frac{c}{I_a} \dot{{y}} + \frac{K_{tb}}{I_a} {y} 
    = \frac{H K_{po}}{I_a} \omega \,.
\end{equation}
If the gyroscope is well-designed, its dynamics will be critically damped with 
a gain close to $1$. This will make $y$ converge quickly to $\omega$ (more 
precisely, $y(t)-\omega(t)\rightarrow 0$, $t\rightarrow 0$ when $\omega$ 
changes slowly or is constant). \\

By introducing 
$a_1 = c/I_a$, $a_2 = K_{tb}/I_a$, $b_1 = H/(I_aK_{po})$,
we can rewrite~\eqref{eq:gyro:harm1} in the ``standard'' form
\begin{equation}
\label{eq:out:gyro}
    \ddot{y} + a_1 \dot{y} + a_2 y = b_1 \dot{u}\,.
\end{equation}
Here we also introduced $\dot{u} = \omega$, which is meant to emphasize that 
the quantity to be estimated is the angular displacement $u$ instead of the 
angular velocity ($\omega$). Now, the right-hand side is critically damped 
with a gain of $1$ if the following holds:
\begin{equation}
\label{eq:cd}
\begin{split}
    \textrm{gain} = \frac{b_1}{a_2} = 1 
    \quad , \quad
    D = a_1^2 - 4 a_2 = 0 .
\end{split}    
\end{equation}
Thus, if we are interested in estimating the parameters, we only need to 
estimate one parameter as the others can then be determined from~\eqref{eq:cd}.

\section{Simulation}

The behavior of the gyro is given by the ODE described by \eqref{eq:out:gyro}
with the initial condition $y(t_0) = y_0$, $\dot{y}(t_0) = \dot{y}_0$,
were $y_0,\dot{y}_0$ are the known initial conditions (usually, they would be
$(0,0)$, assuming a resting robot). \\

In order to simulate this ODE, we first transform it to a (multi-dimensional)
first order ODE. Second, we will need to deal with the issue that 
the angular velocity ($\dot{u}$) is not available, just the angles ($u$). \\
 
To deal with this second issue, we integrate 
($\ref{eq:out:gyro}$) respect to the time and get
\begin{equation}
    \dot{y} + a_1 y + a_2 \int y = b_1 u .
\end{equation}
Next, by introducing  $w_1 = \int y$ and $w_2 = y$, we  rewrite this system as:
\begin{equation} \begin{split}
    \dot{w}_1 &= w_2\,, \\
    \dot{w}_2 &= - a_1 w_2 - a_2 w_1 + b_1 u\,,
\end{split} \end{equation}
where $w_1(t_0) = w_2(t_0) = 0$. This can be further transformed into the
\begin{equation}
    \dot{w} = E w + F u
\end{equation}
first order ODE form, where $w = [w_1,w_2]^T$ and
\begin{equation}
    E = \left[ \begin{array}{cc}
            0 & 1 \\
            -a_2 & -a_1
        \end{array} \right]
        \quad , \quad
        F = 
        \left[ \begin{array}{c}
            0 \\
            b_1
        \end{array} \right] .
\end{equation}

\end{document}

